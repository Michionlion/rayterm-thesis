%
% $Id: conclusion.tex
%
%   *******************************************************************
%   * SEE THE MAIN FILE "AllegThesis.tex" FOR MORE INFORMATION.       *
%   *******************************************************************
%

\chapter{Discussion and Future Work}\label{ch:conclusion}

\section{Summary of Results}\label{ch:conclusion:summary}

Over the many years of innovation in the field of computer graphics, advances in rendering have led to massive increases in the fidelity of engaging, satisfying, and realistic computer visualizations.
With its initial development complete, \name{} is a new and unique entry into the ranks of rendering engines.
It is reminiscent of retro aesthetics of the seventies and eighties but also uses new technology like ray-tracing to deliver compelling possibilities.
\name{} can now be used as an engine for terminal-based 3D tools, visualizations, games, and more.
The current implementation has been completely open-sourced under the AGPL v3 license, and any contributions are welcome.

Throughout the development of \name{} new possibilities were entertained, sometimes discarded, but ultimately kept and distilled to the immature product we have now.
In its current form, \name{} could easily be used to implement a terminal viewer for model files, arguably an incredibly useful tool that does not yet exist.
As \name{} is designed in a simple, open, and accessible format, we hope that future improvements will make \name{} even better, more configurable, and extensible.

\section{Future Work}\label{ch:conclusion:future}

There is a lot of room for improvement in \name{}; it is, after all, an extremely early version of an ambitious idea.
One major improvement possible is the creation of a scene description language.
Currently, the scene must be set programmatically; if the same flexibility was given to scenes as is materials, user application development would be even easier.
Another equally useful improvement would be support for ``Material Definition Language''~\cite{nvidia2015mdl} materials.
This could involve a large reworking of the asset loading system but should provide an easy development process that does not require any CUDA programming knowledge to create high-quality ray-traced scenes in a terminal.

Numerous other improvements are possible, such as environment mapping, explicit lighting, motion blur, triangle animation (which is actually somewhat supported), and much more.
All of these improvements are possible with future collaborators on the open-sourced \name{} implementation.

\section{Conclusion}\label{ch:conclusion:end}

\name{} is a complex system, the implementation and design of which is an ongoing process.
The system currently supports real-time 3D scene viewing through a terminal emulator.
This is made possible through the OptiX library allowing ray-surface intersection calculations to be done a highly parallelized manner on a GPU.
The Tickit interface and its half-pixel translation algorithm, along with \texttt{libtickit} itself, enable \name{} to treat a terminal as an image canvas.
\name{} is now a fully real-time capable ray-tracing engine, rendering 30 or more unicode character images per second to a terminal window on a moderately powerful computer.
Since the initial startup development of \name{} is complete, all contributions to \name{} development are now welcome.
In pursuit of maintaining good open-source readability and ease of development, documentation has been and will continue to be a priority throughout the transition process.
We hope that future projects find \name{} inspiring and useful, whether as an engine or as a jumping-off point for real-time ray-tracing and terminal-based applications.
