%
% $Id: conclusion.tex
%
%   *******************************************************************
%   * SEE THE MAIN FILE "AllegThesis.tex" FOR MORE INFORMATION.       *
%   *******************************************************************
%

\chapter{Discussion and Future Work}\label{ch:conclusion}

\section{Summary of Results}\label{ch:conclusion:summary}

\section{Future Work}\label{ch:conclusion:future}

\begin{enumerate}
  \item more materials / MDL support
  \item scene description language
  \item better rtexplore movement
  \item environment image mapping (missed rays => image lookup)
  \item much more...
\end{enumerate}

\section{Conclusion}\label{ch:conclusion:end}

\name\ is a complex system, the implementation and design of which is an ongoing process.
The system currently supports real-time 3D scene viewing through a terminal emulator.
This is made possible through the OptiX library allowing ray-surface intersection calculations to be done a highly parallelized manner on a GPU.
The Tickit interface and its half-pixel translation algorithm, along with \texttt{libtickit} itself, enable \name\ to treat a terminal as an image canvas.
\name\ is now a fully real-time capable ray-tracing engine, rendering 30 or more unicode character images per second to a terminal window on a moderately powerful computer.
Since the initial startup development of \name\ is complete, all contributions to \name\ development are now welcome.
In pursuit of maintaining good open-source readability and ease of development, documentation has been and will continue to be a priority throughout the transition process.
We hope that future projects find \name\ inspiring and useful, whether as an engine or as a jumping-off point for real-time ray-tracing and terminal-based games.
